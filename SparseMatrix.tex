\documentclass[a4paper,12pt]{article}
\usepackage[utf8]{inputenc}
\usepackage[T1]{fontenc} % Adicione este pacote para suportar acentos corretamente
\usepackage{amsmath}
\usepackage{algorithm}
\usepackage[noend]{algpseudocode}
\usepackage{listings}
\usepackage[portuguese]{babel} % Adicione este pacote para suportar a língua portuguesa
\usepackage{hyperref}

\lstset{
    breaklines=true, % Ativa a quebra de linha automática nos blocos de código
    literate=%
    {á}{{\'a}}1
    {à}{{\`a}}1
    {ã}{{\~a}}1
    {ê}{{\^e}}1
    {é}{{\'e}}1
    {í}{{\'i}}1
    {ó}{{\'o}}1
    {õ}{{\~o}}1
    {ú}{{\'u}}1
    {ç}{{\c{c}}}1 % Configura o caractere "ç"
}

\title{Relatório: Implementação de Matriz Esparsa em C++}
\author{Nauan Aires e Victor Cavalcante}
\date{\today}

\begin{document}

\maketitle

\section{Introdução}

Neste relatório, descrevemos a implementação de uma Matriz Esparsa em C++. Uma matriz esparsa é uma matriz na qual a maioria dos elementos é zero. Para economizar espaço e melhorar a eficiência em operações com matrizes esparsas, utilizamos uma estrutura de dados especializada para representar apenas os elementos não nulos da matriz.
Qualquer informação adicional ou explicações mais diretas e pontuais sobre cada função e método utilizado pode ser consultado através do Github do projeto linkado na seção de Referências \hyperref[sec:ref]{(Clique aqui para ir a seção)} 

\section{Estrutura de Dados}

Para implementar a matriz esparsa, utilizamos uma lista de listas (ou vetor de vetores) para armazenar os elementos não nulos. Cada elemento não nulo é representado por um objeto que contém a posição do elemento na matriz (linha e coluna) e o valor do elemento. A estrutura do projeto foi definida com os seguintes arquivos:
\\"Node.h"
\\"SparseMatrix.h"
\\"main.cpp"
\\E o nó é definido pela seguinte estrutura:
\begin{lstlisting}[language=C++]
struct Node {
    Node *direita;
    Node *abaixo;
    int linha;
    int coluna;
    double valor;

    Node(Node *dir, Node *ab, int l, int c, double v) {
        this->direita = dir;
        this->abaixo = ab;
        this->linha = l;
        this->coluna = c;
        this->valor = v;
    }
};
\end{lstlisting}

Nessa implementação, cada linha da matriz esparsa é representada por um vetor de elementos não nulos. A matriz esparsa em si é um vetor de linhas.

\section{Operações}

Implementamos as principais operações, como a adição de uma matriz por arquivo, a soma de matriz + matriz, multiplicação de matrizes esparsas, criação de um arquivo através de uma matriz inserida e a leitura de um arquivo contendo uma matriz dentro da aplicação.
\\\\
Segue abaixo a listagem e demonstração de cada caso de operação.

\subsection{Caso 1 - Adicionar Matriz}

Para adicionar uma nova matriz (convertida em matriz esparsa pelo código) é necessário que o Usuário digite primeiramente o número de linhas e colunas utilizadas por essa matriz, dado por esta implementação:
\begin{lstlisting}[language=C++]
switch(op){
    case 1: {
    cout << endl << "Criacao da matriz..." << endl;
    int m, n, value;
    cout << "Insira o numero de linhas e colunas: ";
    cin >> m >> n;

    SparseMatrix* matriz = new SparseMatrix(m, n);

    cout << endl;
    matriz->print();
    cout << endl;

    int c {0};
    while(c == 0){
        cout << "Insira a posicao e valor na matriz: ";
        cin >> m >> n >> value;
        matriz->insert(m, n, value);
        cout << endl;
        matriz->print();
        cout << endl;
        cout << "Finalizar?" << endl;
        cout << "1 - Sim" << endl << "0 - Nao" << endl;
        cin >> c;
        cout << endl;
    }

    cout << "Matriz adicionada com sucesso!" << endl << endl;
\end{lstlisting}

Após a adicionar a nova matriz, é dada a opção do usuário salvar a matriz criada em um arquivo, segue abaixo a implementação:
\begin{lstlisting}[language=C++]
    cout << "Matriz adicionada com sucesso!" << endl << endl;
    cout << "Salvar em um arquivo?" << endl;
    cout << "1 - Sim" << endl << "0 - Nao" << endl << endl;
    int key;
    cin >> key;
    if(key == 1){
        cout << "Digite o nome do arquivo para salvar a matriz: ";
        string filename;
        cin >> filename;

        ofstream file(filename);
        if(!file.is_open()){
            throw std::runtime_error("Nao foi possivel abrir o arquivo: " + filename);
        }

        file << matriz->getMaxM() << " " << matriz->getMaxN() << std::endl;

        for(int i = 0; i < matriz->getMaxM(); i++) {
            for(int j = 0; j < matriz->getMaxN(); j++) {
                double value = matriz->get(i, j);
                if(value != 0.0){
                    file << i << " " << j << " " << value << std::endl;
                }
            }
        }

        file.close();

        cout << "Matriz salva com sucesso!" << endl;
    }
        break;
\end{lstlisting}
\subsection{Caso 2 - Somar Matrizes}

A soma das matrizes é representada pela soma de dois arquivos que precisam estar presentes dentro da pasta do projeto, sendo assim possível a soma esses dois arquivos através da seguinte implementação:


\begin{lstlisting}[language=C++]
//Método para ler os arquivos contendo matrizes
SparseMatrix *readSparseMatrix(const std::string &nomeArquivo){
    std::ifstream arquivo(nomeArquivo);
    if(!arquivo.is_open()){
        throw std::runtime_error("Nao foi possivel abrir o arquivo: " + nomeArquivo);
    }

    int m, n;
    arquivo >> m >> n; // Lê as dimensões da matriz

    SparseMatrix *matriz = new SparseMatrix(m, n);

    while(true){
        double value;
        arquivo >> m >> n >> value;
        if(arquivo.eof()) break;
        matriz->insert(m, n, value);
    }

    arquivo.close();

    return matriz;
}

//Método para realizar a soma
SparseMatrix *sum(SparseMatrix *A, SparseMatrix *B){
    // Verifica se as matrizes têm o mesmo tamanho
    if(A->getMaxM() != B->getMaxM() || A->getMaxN() != B->getMaxN()){
        throw std::invalid_argument("As matrizes devem ter o mesmo tamanho para serem somadas.");
    }

    int m = A->getMaxM();
    int n = A->getMaxN();

    // Cria uma nova matriz C com o mesmo tamanho de A e B
    SparseMatrix *C = new SparseMatrix(m, n);
    // Percorre todas as posições da matriz
    for(int i = 0; i < m; i++){
        for(int j = 0; j < n; j++){
            // Obtém os valores de A e B nas posições (i, j)
            double valueA = A->get(i, j);
            double valueB = B->get(i, j);

            // Calcula a soma dos valores e insere na matriz C
            double sum = valueA + valueB;
            C->insert(i, j, sum);
        }
    }

    return C; // Retorna a matriz C resultante da soma de A e B
}

case 2: {
    cout << endl << "Digite o nome do arquivo da primeira matriz: ";
    string filenameA;
    cin >> filenameA;

    SparseMatrix *A = readSparseMatrix(filenameA);
    
    cout << "Digite o nome do arquivo da segunda matriz: ";
    string filenameB;
    cin >> filenameB;

    SparseMatrix *B = readSparseMatrix(filenameB);

    SparseMatrix *C = sum(A, B);
    cout << "Matriz multiplicada com sucesso!" << endl;
    cout << endl;
    C->print();
    cout << endl;
    cout << "Salvar em um arquivo?" << endl;
    cout << "1 - Sim" << endl << "0 - Nao" << endl;
    int key;
    cin >> key;
    if(key == 1){
        cout << "Digite o nome do arquivo para salvar a matriz: ";
        string filename;
        cin >> filename;

        ofstream file(filename);
        if(!file.is_open()){
            throw std::runtime_error("Nao foi possível abrir o arquivo: " + filename);
        }

        file << C->getMaxM() << " " << C->getMaxN() << std::endl;

        for (int i = 0; i < C->getMaxM(); i++) {
            for (int j = 0; j < C->getMaxN(); j++) {
                double value = C->get(i, j);
                if(value != 0.0){
                    file << i << " " << j << " " << value << std::endl;
                }
            }
        }

        file.close();

        cout << "Matriz salva com sucesso!" << endl;
        cout << endl;
    }


    // Liberar memória
    delete A;
    delete B;
    delete C;

    break;
}
\end{lstlisting}

\subsection{Caso 3 - Multiplicar Matrizes}

Neste caso de multiplicação de matrizes é utilizado a mesma função de leitura citado no caso anterior, porém dessa vez utilizaremos um método de multiplicação citado logo abaixo na implementação, e em seguida a exibição e lógica para o caso 3.

\begin{lstlisting}[language=C++]
//Lógica/Método para a função de multiplicação (multiply)
SparseMatrix *multiply(SparseMatrix *A, SparseMatrix *B) {
    if(A->getMaxN() != B->getMaxM()){
        throw std::runtime_error("As matrizes nao podem ser multiplicadas. O numero de colunas de A deve ser igual ao numero de linhas de B.");
    }
    
    int m = A->getMaxM();
    int n = B->getMaxN();
    int p = B->getMaxM();
    
    SparseMatrix *C = new SparseMatrix(m, p);  // C terá dimensões m x p
    
    for(int i = 0; i < m; i++){
        for(int j = 0; j < p; j++){
            int sum = 0;
            for(int k = 0; k < n; k++){
                sum += A->get(i, k) * B->get(k, j);  // Multiplica os elementos correspondentes e soma
            }
            if(sum != 0){
                C->insert(i, j, sum);  // Insere o elemento não nulo na matriz resultante C
            }
        }
    }
    
    return C;
}

case 3: {
    cout << endl << "Digite o nome do arquivo da primeira matriz: ";
    string filenameA;
    cin >> filenameA;

    SparseMatrix *A = readSparseMatrix(filenameA);
    
    cout << "Digite o nome do arquivo da segunda matriz: ";
    string filenameB;
    cin >> filenameB;

    SparseMatrix *B = readSparseMatrix(filenameB);

    SparseMatrix *C = multiply(A, B);
    cout << "Matriz multiplicada com sucesso!" << endl;
    cout << endl;
    C->print();
    cout << endl;
    cout << "Salvar em um arquivo?" << endl;
    cout << "1 - Sim" << endl << "0 - Nao" << endl;
    int key;
    cin >> key;
    if(key == 1){
        cout << "Digite o nome do arquivo para salvar a matriz: ";
        string filename;
        cin >> filename;

        ofstream file(filename);
        if(!file.is_open()){
            throw std::runtime_error("Nao foi possivel abrir o arquivo: " + filename);
        }

        file << C->getMaxM() << " " << C->getMaxN() << std::endl;

        for(int i = 0; i < C->getMaxM(); i++){
            for(int j = 0; j < C->getMaxN(); j++){
                double value = C->get(i, j);
                if(value != 0){
                    file << i << " " << j << " " << value << std::endl;
                }
            }
        }

        file.close();

        cout << "Matriz salva com sucesso!" << endl;
        cout << endl;
    }

    // Liberar memória
    delete A;
    delete B;
    delete C;

    break;
}
\end{lstlisting}

\subsection{Caso 4 - Abrir Arquivo}
O caso 4 tem como função exclusiva apenas ler um arquivo através do nome fornecido, utilizará a mesma função Read dos outros casos, porém será filtrado pelo nome, exemplo: 1- se for fornecido o nome M1, irá ler o arquivo M1, que contém uma matriz pré-definida.
\begin{lstlisting}[language=C++]
case 4: {
    cout << endl << "Digite o nome do arquivo da matriz: ";
    string filenameA;
    cin >> filenameA;
    cout << endl;
    SparseMatrix *A = readSparseMatrix(filenameA);
    A->print();

    break;
}
\end{lstlisting}

\subsection{Caso 5 - Sair}

Esse caso tem como função exclusiva sair e finalizar a aplicação, para que o usuário tenha controle visual e também que seja intuitivo para todos os tipos de usuários. Segue abaixo a implementação:
\begin{lstlisting}[language=C++]
case 5:{
    cout << "Saindo..." << endl;
    return 0;
}
\end{lstlisting}

\section{Resultados}

Testamos nossa implementação utilizando diversas matrizes esparsas de diferentes tamanhos e realizamos operações de Carregar matriz por arquivo, Soma, Multiplicação e Criação de novo texto de matriz. Os resultados foram consistentes e a implementação mostrou-se eficiente para lidar com matrizes esparsas.
\\\\
Todos os testes podem ser consultados através dos arquivos com a extensão .txt anexados a pasta compactada deste mesmo documento.
\\\\
Também é possível ver com mais detalhes as implementações mais detalhada sobre cada função através do arquivo Readme.md no GitHub, ou até mesmo anexado a pasta compactada.

\section{Como o trabalho foi Dividido e Desenvolvido}
O trabalho foi desenvolvido com ajuda do Git e plataforma GitHub, onde \underline{Victor Cavalcante} tinha domínio sobre a ferramenta e o mesmo também queria manter toda a aplicação em seu repositório GitHub e \underline{Nauan} como não dominava a ferramenta e tinha um pouco de dificuldade com o conteúdo, o mesmo baixava os arquivos através do repositório GitHub de Victor, modificava e envia novamente para o mesmo fazer atualizar o código dentro da plataforma, afim de prevenir erros e evitar dor de cabeças com o código, pois o mesmo nunca tinha utilizado.
\\Como \underline{Victor} tinha mais experiência e facilidade com a parte da programação, fez maioria de suas funções e métodos.
\\\underline{Nauan} Fez algumas funções e lógicas mais simples como o Destrutor localizado em SparseMatrix.h, representado por:

\begin{lstlisting}[language=C++]
SparseMatrix::~SparseMatrix(){
    Node *auxLinha = m_head->abaixo;

    while (auxLinha != m_head) {
        Node *auxColuna = auxLinha->direita;
        while (auxColuna != auxLinha) {
            Node *temp = auxColuna;
            auxColuna = auxColuna->direita;
            delete temp;
        }
        Node *temp = auxLinha;
        auxLinha = auxLinha->abaixo;
        delete temp;
    }

    delete m_head;
}
\end{lstlisting}

Também a função Print, para a visualização da Matriz:

\begin{lstlisting}[language=C++]
void SparseMatrix::print(){
    int m = this->maxm; // Armazena o número de linhas da matriz
    int n = this->maxn; // Armazena o número de colunas da matriz

    for (int i = 0; i < m; i++){ // Percorre as linhas da matriz
        for (int j = 0; j < n; j++){ // Percorre as colunas da matriz
            std::cout << get(i, j) << " "; // Obtém o valor da posição (i, j) e imprime na tela
        }
        std::cout << '\n'; // Quebra de linha após imprimir uma linha completa da matriz
    }
}
\end{lstlisting}

A Função de Soma, representada abaixo:

\begin{lstlisting}[language=C++]
SparseMatrix *sum(SparseMatrix *A, SparseMatrix *B){
    // Verifica se as matrizes têm o mesmo tamanho
    if(A->getMaxM() != B->getMaxM() || A->getMaxN() != B->getMaxN()){
        throw std::invalid_argument("As matrizes devem ter o mesmo tamanho para serem somadas.");
    }

    int m = A->getMaxM();
    int n = A->getMaxN();

    // Cria uma nova matriz C com o mesmo tamanho de A e B
    SparseMatrix *C = new SparseMatrix(m, n);
    // Percorre todas as posições da matriz
    for(int i = 0; i < m; i++){
        for(int j = 0; j < n; j++){
            // Obtém os valores de A e B nas posições (i, j)
            double valueA = A->get(i, j);
            double valueB = B->get(i, j);

            // Calcula a soma dos valores e insere na matriz C
            double sum = valueA + valueB;
            C->insert(i, j, sum);
        }
    }

    return C; // Retorna a matriz C resultante da soma de A e B
}
\end{lstlisting}

E também a função de Ler Arquivos, representada abaixo:

\begin{lstlisting}[language=C++]
SparseMatrix *readSparseMatrix(const std::string &nomeArquivo){
    std::ifstream arquivo(nomeArquivo);
    if(!arquivo.is_open()){
        throw std::runtime_error("Nao foi possivel abrir o arquivo: " + nomeArquivo);
    }

    int m, n;
    arquivo >> m >> n; // Lê as dimensões da matriz

    SparseMatrix *matriz = new SparseMatrix(m, n);

    while(true){
        double value;
        arquivo >> m >> n >> value;
        if(arquivo.eof()) break;
        matriz->insert(m, n, value);
    }

    arquivo.close();

    return matriz;
}
\end{lstlisting}

E \underline{Nauan} Fez boa parte do relatório em LaTeX para equilibrar os trabalhos, em relação ao Relatório \underline{Victor} trabalhou em:
\\• Seção 6 - Dificuldade encontradas
\\• Seção 7 - Análise de complexidade de pior caso das funções: insert, get e sum
\\• Seção 8 - Listagem dos testes
\\• Seção 9 - Referências

\section{Dificuldades encontradas}
As principais dificuldades encontradas foram com a função Insert, onde quando ia passar para próxima linha sempre quebrava o código e mostrava apenas zeros. A dificuldade foi resolvida através de um amigo, Gustavo Henrique de (ES) que mora juntamente com Victor.
\\\\
E que por conta desse mesmo problema também foi encontrado problemas no construtor da matriz, que foi resolvido com o mesmo amigo.
\section{Análise de complexidade de pior caso das funções: insert, get e sum}
Esta aqui a implementação do Insert, onde a complecidade de cada ponto estará comentada:
\begin{lstlisting}[language=C++]
void SparseMatrix::insert(int m, int n, double value){
    // Verifica se as coordenadas estão dentro dos limites válidos da matriz
    if (m < 0 || n < 0 || m >= this->maxm || n >= this->maxn){
        throw std::out_of_range("Posição inválida na matriz.");
    }

    // Encontra o nó da linha apropriada para inserção
    Node *linhaH = this->m_head->abaixo;
    
    while (linhaH != m_head && linhaH->linha < m){
        linhaH = linhaH->abaixo;
    }

    // Encontra o nó da coluna apropriada para inserção
    Node *colunaH = m_head->direita;

    while (colunaH != m_head && colunaH->coluna < n){
        colunaH = colunaH->direita;
    }

    // Encontra o nó vizinho à direita na mesma linha
    Node *noVizinhoC = linhaH;

    while (noVizinhoC->direita != linhaH && noVizinhoC->direita->coluna < n){
        noVizinhoC = noVizinhoC->direita;
    }

    // Encontra o nó vizinho abaixo na mesma coluna
    Node *noVizinhoL = colunaH;

    while (noVizinhoL->abaixo != colunaH && noVizinhoL->abaixo->linha < m){
        noVizinhoL = noVizinhoL->abaixo;
    }

    // Se o elemento já existe na posição, atualiza seu valor
    if (noVizinhoL->direita->linha == m && noVizinhoC->abaixo->coluna == n){
        noVizinhoL->valor = value;
        return;
    }

    // Cria um novo nó e o insere na posição apropriada
    Node *novo = new Node(noVizinhoC->direita, noVizinhoL->abaixo, m, n, value);
    noVizinhoC->direita = novo; 
    noVizinhoL->abaixo = novo;
}
\end{lstlisting}
Verificação das coordenadas: Essa parte possui uma complexidade constante O(1), pois envolve apenas comparações e verificações simples.
\\\\
Busca do nó da linha apropriada: A complexidade depende do número de linhas presentes antes da linha desejada. No pior caso, em uma matriz esparsa completamente preenchida, a busca percorrerá todas as linhas da matriz até encontrar a posição correta. Portanto, a complexidade dessa parte é O(n), onde n é o número total de linhas da matriz.
\\\\
Busca do nó da coluna apropriada: De maneira similar à busca da linha, a complexidade depende do número de colunas presentes antes da coluna desejada. No pior caso, em uma matriz esparsa completamente preenchida, a busca percorrerá todas as colunas da matriz até encontrar a posição correta. Assim, a complexidade dessa parte é O(m), onde m é o número total de colunas da matriz.
\\\\
Busca do nó vizinho à direita na mesma linha: Essa busca percorre os nós da linha até encontrar a posição correta. No pior caso, onde todos os elementos da linha estão preenchidos, a busca percorrerá todos os nós da linha. Portanto, a complexidade é O(n), onde n é o número total de colunas da matriz.
\\\\
Busca do nó vizinho abaixo na mesma coluna: De maneira similar à busca na linha, essa busca percorre os nós da coluna até encontrar a posição correta. No pior caso, onde todos os elementos da coluna estão preenchidos, a busca percorrerá todos os nós da coluna. Assim, a complexidade é O(m), onde m é o número total de linhas da matriz.
\\\\
Verificação se o elemento já existe: Essa verificação envolve apenas comparações de valores de nós, portanto, possui uma complexidade constante O(1).
\\\\
Inserção de um novo nó: Essa parte envolve a criação de um novo nó e a atualização dos ponteiros dos nós vizinhos. Portanto, possui uma complexidade constante O(1).
\\\\
Segue abaixo a análise da função get():
\begin{lstlisting}[language=C++]
double SparseMatrix::get(int m, int n){
    // Verifica se as coordenadas estão dentro dos limites válidos da matriz
    if (m < 0 || n < 0 || m >= this->maxm || n >= this->maxn){
        throw std::out_of_range("Posicao invalida na matriz.");
    }

    // Encontra o nó da linha correta na lista encadeada
    Node *linhaH = m_head->abaixo;

    while (linhaH != m_head && linhaH->linha < m){
        linhaH = linhaH->abaixo;
    }

    // Verifica se a linha correta foi encontrada
    if (linhaH != m_head && linhaH->linha == m){
        Node *colunaH = linhaH->direita;
        // Percorre os nós na linha até encontrar a coluna desejada
        while (colunaH != linhaH && colunaH->coluna != n){
            colunaH = colunaH->direita;
        }

        // Verifica se a coluna correta foi encontrada
        if (colunaH != linhaH && colunaH->coluna == n){
            return colunaH->valor; // Retorna o valor do nó
        }
    }

    return 0; // Retorna 0 se não foi encontrado nenhum valor na posição especificada
}
\end{lstlisting}
Verificação das coordenadas: Essa parte possui uma complexidade constante O(1), pois envolve apenas comparações e verificações simples.
\\\\
Busca do nó da linha correta: A complexidade depende do número de linhas presentes antes da linha desejada. No pior caso, em uma matriz esparsa completamente preenchida, a busca percorrerá todas as linhas da matriz até encontrar a posição correta. Portanto, a complexidade dessa parte é O(n), onde n é o número total de linhas da matriz.
\\\\
Verificação da existência da linha correta: Essa verificação envolve apenas comparações de valores de nós, portanto, possui uma complexidade constante O(1).
\\\\
Busca do nó da coluna correta: Após encontrar a linha correta, essa busca percorre os nós na linha até encontrar a coluna desejada. No pior caso, onde todos os elementos da linha estão preenchidos, a busca percorrerá todos os nós da linha. Portanto, a complexidade é O(n), onde n é o número total de colunas da matriz.
\\\\
Verificação da existência da coluna correta: Essa verificação envolve apenas comparações de valores de nós, portanto, possui uma complexidade constante O(1).
\\\\
Segue abaixo a análise da função sum(soma):
\begin{lstlisting}[language=C++]
SparseMatrix *sum(SparseMatrix *A, SparseMatrix *B){
    // Verifica se as matrizes têm o mesmo tamanho
    if(A->getMaxM() != B->getMaxM() || A->getMaxN() != B->getMaxN()){
        throw std::invalid_argument("As matrizes devem ter o mesmo tamanho para serem somadas.");
    }

    int m = A->getMaxM();
    int n = A->getMaxN();

    // Cria uma nova matriz C com o mesmo tamanho de A e B
    SparseMatrix *C = new SparseMatrix(m, n);
    // Percorre todas as posições da matriz
    for(int i = 0; i < m; i++){
        for(int j = 0; j < n; j++){
            // Obtém os valores de A e B nas posições (i, j)
            double valueA = A->get(i, j);
            double valueB = B->get(i, j);

            // Calcula a soma dos valores e insere na matriz C
            double sum = valueA + valueB;
            C->insert(i, j, sum);
        }
    }

    return C; // Retorna a matriz C resultante da soma de A e B
}
\end{lstlisting}
Verificação do tamanho das matrizes: Essa parte envolve comparações dos tamanhos das matrizes A e B. Como essas informações são obtidas diretamente através dos métodos getMaxM() e getMaxN(), a complexidade é O(1), ou seja, constante.
\\\\
Criação da matriz C: A criação da matriz C envolve a alocação de memória para a nova matriz e a inicialização de seus atributos de tamanho. Essa operação possui uma complexidade constante O(1).
\\\\
Percorrendo todas as posições das matrizes A e B: Essa parte envolve dois loops aninhados que percorrem todas as posições das matrizes A e B. O número de iterações é determinado pelo tamanho das matrizes, m e n. Portanto, a complexidade dessa parte é O(m * n), onde m é o número de linhas e n é o número de colunas.
\\\\
Obtendo os valores de A e B e calculando a soma: Essa operação envolve a chamada dos métodos get() para obter os valores das matrizes A e B nas posições (i, j), e a soma dos valores obtidos. Essas operações têm uma complexidade constante O(1).
\\\\
Inserção do valor calculado na matriz C: A inserção do valor calculado na matriz C envolve a chamada do método insert(), que tem uma complexidade constante O(1).
\section{Listagem dos testes}
Todos os testes realizados foram utilizando os arquivos txt anexados junto ao projeto. Segue abaixo a pré visualização dos dois arquivos:
\\\\
M1.txt
\begin{lstlisting}[language=C++]
3 3
0 0 1
0 1 2
0 2 3
1 0 4
1 1 5
1 2 6
2 0 7
2 1 8
2 2 9
\end{lstlisting}
M2.txt
\begin{lstlisting}[language=C++]
3 3
0 0 1
0 1 1
0 2 1
1 0 1
1 1 1
1 2 1
2 0 1
2 1 1
2 2 1
\end{lstlisting}
soma.txt
\begin{lstlisting}[language=C++]
3 3
0 0 2
0 1 3
0 2 4
1 0 5
1 1 6
1 2 7
2 0 8
2 1 9
2 2 10
\end{lstlisting}
\section{Referências}\label{sec:ref}
1. Minicurso de Latex \href{https://www.youtube.com/playlist?list=PLjbaYg8NJsoNZY87IDxKzZ6x2XeXtICim}{(Link aqui)}
\\2. Vídeo para auxiliar construção e lógica inicial do projeto \href{https://www.youtube.com/watch?v=tMoJYQ3pkUw&t}{(Link aqui)}
\\3. Projeto concluído no GitHub e acesso a informações extras \href{https://www.youtube.com/watch?v=tMoJYQ3pkUw&t}{(Link aqui)}
\\4. Repositório GitHub que utilizamos para nos orientar \href{https://github.com/ArthurAssuncao/Matriz_Esparsa}{(Link aqui)}
\\5. Ferramenta de I.A utilizada para auxiliar a encontrar erros e soluções no código e relatório LaTeX \href{https://chat.openai.com/}{(Link aqui)}
\section{Conclusão}
Neste trabalho, apresentamos a implementação de uma Matriz Esparsa em C++ utilizando uma estrutura de dados especializada. A implementação mostrou-se eficiente para lidar com matrizes esparsas e permite realizar as principais operações de forma adequada. A utilização de uma estrutura de dados especializada para matrizes esparsas é fundamental para economizar espaço e melhorar a eficiência em operações com matrizes esparsas.
\\\\
Qualquer dúvida ou informação adicional pode ser consultada através do Github do projeto (link listado acima).
\end{document}